\documentclass{article}
\author{Desmond Tuiyot}
\title{Dance App development notes/diary}
\begin{document}
\maketitle
\section{01/15/2020}
\begin{itemize}
	\item \textbf{Multi-window issue:} I had the idea to merge move\_suggest\_code and move\_challenge\_code since they depend on each other. User should be able to move from one window to the next and vice versa.\\
	Decided to have 2 classes in one file. The class objects are instantiated globally and are hidden and shown as means of opening and closing them.
	\item \textbf{Start, Pause and Stop suggestions: }This one was interesting. I created a daemon thread that creates an Event that updates the text display every x seconds. I made it an event so that it would be possible to stop the Event using the Event.set() method to set a flag that the Event will be looking out for. Added new states for the buttons: \texttt{Start} becomes \texttt{Restart} and vice versa, \texttt{Pause} becomes \texttt{Resume} and vice versa, \texttt{Pause} is disabled initially and on \texttt{Stop}.    
\end{itemize}
\textbf{Playlist}
\begin{itemize}
	\item Roddy Ricch - Moonwalking
	\item Roddy Ricch - Gods Eyes
	\item Roddy Ricch - High Fashion
	\item Roddy Ricch - War Baby
\end{itemize}

\section{01/16/2020}
\textbf{Agenda}
\begin{itemize}
	\item Think about how to tie the windows together. Make it a cohesive application.
	\item Go through the user interface and check to see that all buttons do what they are supposed to do. 
	\item Look for inspiration to make the app look a little bit better. 
\end{itemize}
\textbf{What I Dealt With}
\begin{itemize}
	\item Did input validation for the \texttt{add\_dance\_code}
\end{itemize}
\textbf{Playlist}
\begin{itemize}
	\item Outkast - Rosa Parks
	\item Outkast - Skew It on the Bar-B
	\item Outkast - Da Art of Storytellin' (Pt. 2)
	\item Outkast - Chonkyfire (where else is this sound)
	\item Kendrick Lamar - Wesley's Theory
	\item Kendrick Lamar - King Kunta
\end{itemize}

\section{01/18/2020}
\begin{itemize}
	\item \texttt{TODO:} Visual feedback about whether the input is valid - this involved changing the shade of the widget (in this case a QLineEdit) accordingly. We want it to be unobstrusive, so lower contrasts are preferred.
	\item \texttt{TODO: } Separating tags when user presses 'Enter'. This could be interesting. 
\end{itemize}
\textbf{Playlist}
\begin{itemize}
	\item Eminem - Marsh
	\item Eminem - Premonition
	\item Eminem - Unaccomodating
	\item Eminem - Yah Yah
\end{itemize}

\section{01/19/2020}
\begin{itemize}
	\item \texttt{TODO:} Visual feedback about whether the input is valid - this involved changing the shade of the widget (in this case a QLineEdit) accordingly. We want it to be unobstrusive, so lower contrasts are preferred.\\
	\textbf{Solution: }Decided to go with a message box instead of changing background color. Could probably implement this in future anyways.
	\item \texttt{TODO: } Separating tags when user presses 'Enter'. This could be interesting. 
	\item \textbf{Main Window: }Created a main window from which the user navigates to all other windows.  
	\item \textbf{Revisiting Modify Dance Window: }\texttt{Tags} and \texttt{Description} are most likely gonna be too long to display in columns. What to do, then? I can add a miniature version of \texttt{Add Dance Window} right below the table view. I'll make the actual table view immutable, and the user can edit using the new field below the table view.\\
	\textbf{Solution: }Ended up adding the form to the side of the table view.  
	\item \texttt{TODO: }Update the dance data fields according to what move is selected. \textbf{Idea: }Move the DanceMove class out into its own module. And import that into each module I have. So I have access to all that stuff. \textbf{DONE}
	\item \textbf{Selection Signals: }Used \texttt{view.setSelectionModel} to access \texttt{selectionChanged} signal. So now both click selections and cursor selections work. \\
	\texttt{TODO: }Make sure that name is always selected.
	\item \textbf{Changing tags edit: }Make it a line edit that adds into a list view on its side. Listen for enter key and pop it into the list view. \textbf{DONE}
	\item \textbf{BIG }\texttt{TODO: }Context menu. Right clicks and shit.
\end{itemize}
\textbf{Playlist}
\begin{itemize}
	\item Mac Miller - Circles
	\item Mac Miller - Good News
	\item Davido, Popcaan - Risky
	\item Lil Wayne - Hustlin'
	\item Lil Wayne - Canon AMG Remix
	\item Pusha T, Kehlani - Retribution
	\item Pusha T - M.F.T.R
	\item Pusha T - Got Em Covered
	\item Pusha T - Intro
	\item Pusha T - Untouchable
	\item Grip - He is ... I am
	\item Eminem - Lock It Up
\end{itemize}

\section{01/20/2020}
\begin{itemize}
	\item \textbf{Changing tags edit: }Changed it to an editable combobox.\\
	\textbf{TODO: }Think about what signals to look out for. Enter and probably textChanged. Then split at commas
\end{itemize}
\begin{itemize}
	\item Travis Scott, ROSALIA, Lil Baby - HIGHEST IN THE ROOM
	\item JACKBOYS, Sheck Wes - GANG GANG 
\end{itemize}
\end{document}